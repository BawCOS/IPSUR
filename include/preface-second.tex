%    IPSUR: Introduction to Probability and Statistics Using R
%    Copyright (C)  2013 G. Jay Kerns
%
%    This file is part of IPSUR.
%
%    Permission is granted to copy, distribute and/or modify this document
%    under the terms of the GNU Free Documentation License, Version 1.3
%    or any later version published by the Free Software Foundation;
%    with no Invariant Sections, no Front-Cover Texts, and no Back-Cover Texts.
%    A copy of the license is contained in the LICENSE file in this 
%    directory.

\cleardoublepage
\phantomsection
\addcontentsline{toc}{chapter}{Preface to the Second Edition}

\chapter*{Preface to the Second Edition}

What is new in the Second Edition?  Almost everything.  I have addressed two of the goals from the first edition.  I have now converted most of the plots to \texttt{ggplot2} format. 

The Second Edition marks a departure from LyX to Emacs Org-Mode.  I went with Org-Mode for many reasons.  I liked LyX, and LyX is definitely a more user-friendly approach to writing a free book.  My workflow, however, has radically changed over the last two years, and I've converted to using Org-Mode for (almost) everything.  It truly is ``Your Life in Plain Text''.

An advantage of the Org-Mode approach is that I can generate an HTML version (that even looks good, still) with a few keystrokes.  That means I can post an HTML version of IPSUR, which I've done.  

The HTML version is \textbf{very} important, for \textbf{more} than the following reasons:  1) a person can read IPSUR without need to do anything else, period, 2) automatic full-text indexing by Google, Bing, YaCY, etc., and, most importantly to me, 3) \textbf{automatic translation to over 40 languages at the click of a button} (with Google Translate, which comes for free with Google Chrome/Chromium).


\section*{Acknowledgements}

The success of the Second Edition (if any) would be due in no small part to the successes of the First Edition, so it would be apropos to copy-paste the acknowledgements from the earlier Preface here.

I think, though, that the \emph{failures} of the First Edition have played an important role as well. I would like to extend gracious thanks to Mr.\ P.J.C.\ Dwarshuis (Hans), Statistician, from The Hague, Netherlands, and J\'{e}{}sus Juan, who, armed with a sharp eye, have pointed out mistakes, misstatements, and places where better discussion is warranted.  It is the selfless contributions of people just like these gentlemen which make the hours spent polishing a FREE book all the more worthwhile.