%    IPSUR: Introduction to Probability and Statistics Using R
%    Copyright (C)  2013 G. Jay Kerns
%
%    This file is part of IPSUR.
%
%    Permission is granted to copy, distribute and/or modify this document
%    under the terms of the GNU Free Documentation License, Version 1.3
%    or any later version published by the Free Software Foundation;
%    with no Invariant Sections, no Front-Cover Texts, and no Back-Cover Texts.
%    A copy of the license is contained in the LICENSE file in this 
%    directory.

\subtitle{Second Edition}

\usepackage{lmodern}
\renewcommand{\sfdefault}{lmss}
\renewcommand{\ttdefault}{lmtt}

% needed packages
\usepackage{amsmath}
\usepackage{amssymb}
\usepackage{amsthm}
\usepackage[english]{babel}
\usepackage{epsfig}
\usepackage{fancyvrb}
\usepackage{fixltx2e}
\usepackage{float}
%\usepackage{floatflt}
\usepackage[T1]{fontenc}
\usepackage{footnote}
%\usepackage{graphics}
\usepackage{graphicx}
\usepackage[utf8]{inputenc}
\usepackage{latexsym}
\usepackage{longtable}
\usepackage{makeidx}
\usepackage{marvosym}
\usepackage{multicol}
%\usepackage{pslatex}
%\usepackage{showidx}
\usepackage{soul}
\usepackage{srcltx}
\usepackage{stmaryrd}
\usepackage{subfig}
\usepackage{textcomp}
%\usepackage{theorem}
\usepackage[subfigure]{tocloft}
\usepackage{txfonts}
\usepackage{upgreek}
\usepackage{url}
\usepackage{varioref}
\usepackage{verbatim}
%\usepackage{wasysym}
\usepackage{wrapfig}


% Page setup
\usepackage[paperwidth=7.44in,paperheight=9.69in]{geometry}
\geometry{verbose,tmargin=1in,bmargin=1in,outer=0.25in,inner=0.75in}
\pagestyle{headings}
\setcounter{secnumdepth}{2}
\setcounter{tocdepth}{1}

\makeindex

% PDF settings
\usepackage[hyperref,x11names]{xcolor}
\usepackage[	unicode=true, 
		bookmarks=true, 
		bookmarksnumbered=true, 
		bookmarksopen=true, 
		bookmarksopenlevel=0, 
		breaklinks=true,
		pdfborder={0 0 0},
		backref=page,
		colorlinks=true]{hyperref}
\hypersetup{pdftitle={Introduction to Probability and Statistics Using R},
 		pdfauthor={G. Jay Kerns}, 
		linkcolor=Firebrick4, 
		citecolor=black, 
		urlcolor=SteelBlue4}

% Listings setup
\usepackage{color}
\usepackage{listings}
\lstset{basicstyle={\ttfamily},
	language=R,
	breaklines=true,
	breakatwhitespace=true,
	keywordstyle={\ttfamily},
	numberstyle = {\ttfamily},
	morestring=[b]"
}




%%%%%%%%%%%%%%%%%%%%%%%%%%%%%% LyX specific LaTeX commands.
\providecommand{\LyX}{L\kern-.1667em\lower.25em\hbox{Y}\kern-.125emX\@}
\newcommand{\noun}[1]{\textsc{#1}}
%% Because html converters don't know tabularnewline
\providecommand{\tabularnewline}{\\}

% special logos
\providecommand{\IPSUR}
{\textsc{I\kern 0ex\lower-0.3ex\hbox{\small P}\kern -0.5ex\lower0.4ex\hbox{\footnotesize S}\kern -0.25exU}\kern -0.1ex\lower 0.15ex\hbox{\textsf{\large R}}\@}

%  user defined commands
% special operators

\renewcommand{\vec}[1]{\mbox{\boldmath$#1$}}

\makeatletter

%%%%%%%%%%%%%%%%%%%%%%%%%%%%%% Textclass specific LaTeX commands.

\numberwithin{equation}{chapter}
\numberwithin{figure}{chapter}

\theoremstyle{plain}
  \newtheorem{thm}{Theorem}[chapter]
  \newtheorem{fact}[thm]{Fact}
  \newtheorem{ax}[thm]{Axiom}
  \newtheorem{prop}[thm]{Proposition}
  \newtheorem{cor}[thm]{Corollary}
  \newtheorem{assumption}[thm]{Assumption}

\theoremstyle{definition}
  \newtheorem{defn}[thm]{Definition}
  \newtheorem{example}[thm]{Example}
  \newtheorem{xca}{Exercise}[chapter]

\theoremstyle{remark}
  \newtheorem{note}[thm]{Note}
  \newtheorem{rem}[thm]{Remark}

\setlength{\cftfignumwidth}{1.5cm}

\@ifundefined{showcaptionsetup}{}{%
 \PassOptionsToPackage{caption=false}{subfig}}
\usepackage{subfig}
\AtBeginDocument{
  \def\labelitemii{\(\circ\)}
}

\makeatother


\newenvironment{exampletoo}{\begin{example}}{\end{example}}

